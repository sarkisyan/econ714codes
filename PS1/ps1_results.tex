% !TeX spellcheck = en_US
\documentclass[11pt,a4paper]{article}
\usepackage[english]{babel}
\usepackage[utf8]{inputenc}
\usepackage[fleqn]{amsmath}
\usepackage{indentfirst}
\usepackage{setspace}
\doublespacing
\usepackage{tabularx}
\usepackage{amsfonts}
\usepackage{dsfont}
\usepackage{caption}
\usepackage{tikz}
\usepackage{amssymb}
%\numberwithin{equation}{section}
\usepackage{graphicx}
\usepackage[bottom]{footmisc}
\usepackage{color}
\allowdisplaybreaks
\usepackage{cite}
%\usepackage{hyperref}
\usepackage{placeins}
\usepackage{titling}
\setlength{\droptitle}{-10em}
\usepackage{booktabs} 
\usepackage{authblk}
\newcommand{\ra}[1]{\renewcommand{\arraystretch}{#1}}
%\hypersetup{pdfstartview=FitH,  linkcolor=linkcolor,urlcolor=urlcolor, colorlinks=true}
\newtheorem{theorem}{Theorem}
\newtheorem{corollary}{Corollary}
\newtheorem{lemma}{Lemma}
\usepackage[colorlinks=true,urlcolor=blue,citecolor=blue,linkcolor=black,bookmarks=true,pdfstartview=FitH]{hyperref}
\usepackage{natbib}
\setcitestyle{round, comma, numbers,sort&compress, super}
\bibliographystyle{chicago}
\textwidth=16cm
\oddsidemargin=0pt
\topmargin = -20pt
\footskip = 20pt
\textheight = 700pt
\textwidth = 450pt
\title{Problem Set 1}
\author{Sergey Sarkisyan}
\date{\today}

\begin{document}
	\maketitle
	\section*{Problem 1}
	\noindent The link to the repository is \href{https://github.com/sarkisyan/econ714codes}{here}.
	
	\section*{Problem 2}
	\noindent The code is in the "ps1\_integration.R" file. The following table shows computation time and result of the integration:\\
	
	\begin{tabular}{|c|c|c|}
		\hline
		Method & Result & Time (sec.) \\
		\hline
		Midpoint & -18.20953 & 0.025 \\
		\hline
		Trapezoid & -18.20953 & 0.015 \\
		\hline
		Simpson & -18.20953 & 0.022 \\
		\hline
		Monte Carlo & -17.98814 & 0.003 \\
		\hline
	\end{tabular}
	\\ \\
	We can see that quadrature methods are more exact but also slower than Monte Carlo. Monte Carlo result also depends on the seed. 
	
	\section*{Problem 3}
	\noindent The code is in the "ps1\_optimization.R" file. The following table shows computation time, optimal $x$ and $y$, and optimal value:\\
	
	\begin{tabular}{|c|c|c|c|}
		\hline
		Method & $x$ & $y$ & Time (sec.)   \\
		\hline
		Newton-Raphson & 1 & 1 & 0.131 \\
		\hline
		BFGS & 1 & 1 & 0.042  \\
		\hline
		Steepest descent & 0.956 & 0.913 & 0.032  \\
		\hline
		Conjugate gradient & 0.956 & 0.912 & 0.034  \\
		\hline
	\end{tabular}
	\\ \\
	Steepest descent and conjugate gradient methods are faster but less precise than Newton-Raphson and BFGS. 
	
	\section*{Problem 4}
	\noindent The code is in the "ps1\_pareto.R" file. I first compute Pareto-efficient distribution for homogenous case. Initial values are equal distribution. Parameter values (identical for all agents in this case) can be found in the code. I use NLOPT package. Specifically, I use ISRES method for optimization. I then introduce heterogeneity and do the same thing for 3 agents and 3 goods. Finally, I solve for 10 homogeneous agents and 10 goods. The optimization is subject to the resource constraint. 
	
	Results for the homogeneous case are the following:\\
	
	\begin{tabular}{|c|c|c|c|c|c|c|c|c|c|}
		\hline
		$x_1^1$ & $x_2^1$ & $x_3^1$ & $x_1^2$ & $x_2^2$ & $x_3^2$ & $x_1^3$ & $x_2^3$ & $x_3^3$ & Time (sec.) \\
		\hline
		1.4035 & 1.3343 & 1.2993& 1.4283 & 1.3426 & 1.2652& 1.4086 & 1.327 & 1.2908 & 5.42 \\
		\hline
	\end{tabular}
	\\ \\
	As expected, all agents get the same quantities of each good. The sum of goods is equal to the sum of endowments. 
	
	Results for the heteregeneous case are the following:\\
	
	\begin{tabular}{|c|c|c|c|c|c|c|c|c|c|}
		\hline
		$x_1^1$ & $x_2^1$ & $x_3^1$ & $x_1^2$ & $x_2^2$ & $x_3^2$ & $x_1^3$ & $x_2^3$ & $x_3^3$ & Time (sec.) \\
		\hline
		0.96702& 0.2981 & 0.6235 & 0.2492& 0.2723& 0.0645 & 3.0914 & 4.6004 & 1.9337& 5.37 \\
		\hline
	\end{tabular}
	\\ \\
	Computation for 10 agents and 10 goods takes 49 seconds. I don't display results here. 
	
	\section*{Problem 5}
	\noindent The code is in the "ps1\_price.R" file. I compute prices by solving the system of 6 equations and 6 unknowns. Unknowns are prices and Lagrande multipliers. Equations are budget constraints and excess demands. I solve equations by minimizing the sum of squares. The prices are 0.010, 0.012, and 0.018. The time is 37 seconds. 
	
\end{document}














